
\documentclass[11pt]{article}
\usepackage[margin=0.5in]{geometry}
\usepackage[utf8]{inputenc}
\usepackage{setspace}
\setstretch{1}
\usepackage[english]{babel}
\usepackage[]{amssymb} 
%\usepackage[enable]{darkmode} 
\usepackage{amsmath, amsthm}
\usepackage{mdframed}
\usepackage{pgfplots}
\usepackage{booktabs}
\usepackage{enumitem}
\usepackage{hyperref}
\usetikzlibrary{pgfplots.fillbetween}  
\pgfplotsset{compat=1.17} 
\makeatletter
\newcommand{\tpmod}[1]{{\@displayfalse\pmod{#1}}}
\makeatother

\mdfdefinestyle{problemstyle}{
    innertopmargin=10pt,
    innerbottommargin=10pt,
    innerrightmargin=10pt,
    innerleftmargin=10pt,
    outerlinewidth=1pt,
    topline=true,
    bottomline=true,
    leftline=true,
    rightline=true
}

%custom enviroment for exercises
\newenvironment{exercise}{
    \begin{mdframed}[style=problemstyle]\textcolor{black}{}
}{
    \end{mdframed}
}


\title{TMATH 126 Week 5 Workshop}
\author{}
\date{\vspace{-5ex}}

\begin{document}
\maketitle
\vspace{-5ex}%not a fan of this, but good fix for now

\begin{exercise}
    A particle moves along a path in space described by the vector-valued function
    $$\vec{r}(t) = \langle 3cost, 3sint, t^{2} \rangle , \quad 0\le t \le 2\pi $$
    \begin{enumerate}[label={\alph*}]
        \item Describe the motion of the particle in words. What shape does the 
            projection of the path make in the $xy$-plane?
        \item Find the tangent vector $\vec{r}\,'(t)$ and compute the arc length 
            of the curve from $t=0$ to $t=3\pi$ (set up the integral and evalute 
            with technology)
        \item Compute the curvature $\kappa (t)$ of the curve. What happens to the
            curvature as $t$ increases?

        \item Now let $z = f(x,y) = x^{2}+y^{2}$, and imagine the movement of the
        particle. What does this look like geometrically.
        \item Find the tangent plane to the surface $z = f(x,y) =x^2+y^2$ at the point where $t=0$.
        \item Use a linear approximation to estimate  the value of $f(x,y) = x^2 + y^2$ 
            near $(x,y)=(2.9,0.1)$.(use the tangent plane found previously) 
    \end{enumerate}
\end{exercise}



\end{document}

