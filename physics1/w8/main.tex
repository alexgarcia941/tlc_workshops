\documentclass[11pt]{article}
\usepackage[margin=0.5in]{geometry}
\usepackage[utf8]{inputenc}
\usepackage{setspace}
\setstretch{1}
\usepackage[english]{babel}
\usepackage[]{amssymb} 
%\usepackage[enable]{darkmode} 
\usepackage{amsmath, amsthm}
\usepackage{mdframed}
\usepackage{pgfplots}
\usepackage{booktabs}
\usepackage{enumitem}
\usepackage{hyperref}
\usetikzlibrary{pgfplots.fillbetween}  
\pgfplotsset{compat=1.17} 
\makeatletter
\newcommand{\tpmod}[1]{{\@displayfalse\pmod{#1}}}
\makeatother

\mdfdefinestyle{problemstyle}{
    innertopmargin=10pt,
    innerbottommargin=10pt,
    innerrightmargin=10pt,
    innerleftmargin=10pt,
    outerlinewidth=1pt,
    topline=true,
    bottomline=true,
    leftline=true,
    rightline=true
}

%custom enviroment for exercises
\newenvironment{exercise}{
    \begin{mdframed}[style=problemstyle]\textcolor{black}{}
}{
    \end{mdframed}
}


\title{TPHYS 121 Workshop Week 8}
\author{}
\date{\vspace{-15ex}}

\begin{document}
\maketitle

\section*{Module 4 Problems}
\subsection*{Exercise 1}
For each of the following let $g=10m/s^2$ unless otherwise stated.
\begin{exercise}
    A $3kg$ box is pushed along a frictionless horizontal surface by a 
    constant force of $12N$. The force is applied in the direction of 
    motion over a distance of $5$ meters.(assume it starts at rest)
    \begin{enumerate}[label=\alph*]
        \item How much work is done on the box? 
        \item What is the velocity of the box after moving $5$ meters?
    \end{enumerate}
\end{exercise}

\subsection*{Exercise 2}
\begin{exercise}
    A $4kg$ object is initially moving at $6m/s$ on a rough horizontal 
    surface. A $15N$ force is applied in the direction of motion over a 
    distance of $4$ meters. The coefficient of kinetic friction between the
    object and the ground is $0.125$
    \begin{enumerate}[label=\alph*]
        \item Calculate the total work done on the object. 
        \item Determine the object's final velocity.
    \end{enumerate}
\end{exercise}

\subsection*{Exercise 3}
\begin{exercise}
    A $6kg$ block slides on a frictionless surface at $4m/s$. It collides
    and sticks to a $4kg$ block that was intitially at rest.
    \begin{enumerate}
        \item Find the velocity of both blocks after the collision. 
        \item Is energy conserved? Why or why not?
    \end{enumerate}
\end{exercise}

\subsection*{Exercise 4}
\begin{exercise}
    A $5kg$ box is pushed up a frictionless ramp inclined at $30^\cdot$ 
    with a constant force of $40N$ applied parallel to the ramp. The box
    starts from rest and moves $3$ meters along the ramp.
    \begin{enumerate}
        \item How much work is done by the applied force? 
        \item How much work is done by the force of gravity?
        \item What is the velocity of the box after moving $3$ meters?
    \end{enumerate}
\end{exercise}

\section*{Additional Resources}
\begin{itemize}
    \item Flipping Physics on youtube: 
        \url{https://www.youtube.com/user/flippingphysics} 
\end{itemize}
\end{document}
