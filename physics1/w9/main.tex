\documentclass[11pt]{article}
\usepackage[margin=0.5in]{geometry}
\usepackage[utf8]{inputenc}
\usepackage{setspace}
\setstretch{1}
\usepackage[english]{babel}
\usepackage[]{amssymb} 
%\usepackage[enable]{darkmode} 
\usepackage{amsmath, amsthm}
\usepackage{mdframed}
\usepackage{pgfplots}
\usepackage{booktabs}
\usepackage{enumitem}
\usepackage{hyperref}
\usetikzlibrary{pgfplots.fillbetween}  
\pgfplotsset{compat=1.17} 
\makeatletter
\newcommand{\tpmod}[1]{{\@displayfalse\pmod{#1}}}
\makeatother

\mdfdefinestyle{problemstyle}{
    innertopmargin=10pt,
    innerbottommargin=10pt,
    innerrightmargin=10pt,
    innerleftmargin=10pt,
    outerlinewidth=1pt,
    topline=true,
    bottomline=true,
    leftline=true,
    rightline=true
}

%custom enviroment for exercises
\newenvironment{exercise}{
    \begin{mdframed}[style=problemstyle]\textcolor{black}{}
}{
    \end{mdframed}
}


\title{TPHYS 121 Workshop Week 9}
\author{}
\date{\vspace{-15ex}}

\begin{document}
\maketitle

\section*{Module 5 Problems}
\subsection*{Exercise 1}
\begin{exercise}
    A solid disk with mass $m=4kg$ and radius $r=0.5m$ is mounted on
    a frictionless axle.($I_{disk} = \frac{1}{2}mr^2$)
    \begin{enumerate}[label=(\alph*)]
        \item Calculate the moment of inerta of the disk about it's central
            axis.
        \item A force of $8N$ us applied tangentially at the rim of the disk
            counterclockwise. Find the resulting torque.
        \item Determine the angular acceleration of the disk.
        \item Using the Right-Hand Rule, determine whether the torque 
            vector points into or out of the page.
    \end{enumerate}
\end{exercise}

\subsection*{Exercise 2}
\begin{exercise}
    A hollow cylinder(a thin-walled hoop) with a mass $2kg$ and radius
    $0.4m$ is initially at rest. A force of $5N$ is applied tangentially 
    to the rim for $3.0s$($I_{hoop} = mr^2$)
    \begin{enumerate}[label=(\alph*)]
        \item Find the moment of inertia of the hoop about it's central axis. 
        \item Determine the angular acceleration of the hoop. 
        \item Calculate the final angular velocity after $3$ seconds.
        \item Find the rotational kinetic energy of the hoop at $3$ seconds.
    \end{enumerate}
\end{exercise}

\subsection*{Exercise 3}
\begin{exercise}
    An ice skater is spinning with an initial angular velocity of $2rad/s$
    and a moment of inertia of $I_1 = 3 kg\cdot m^2$. They pull their arms 
    in reducing their moment of inertia to $I_2 = 1.2 kg\cdot m^2$.
    (Angular momentum: $L=I\omega$)
    \begin{enumerate}[label=(\alph*)]
        \item What happens to the skaters angular speed after pulling 
            their arms in? Why?
        \item Find the skaters new angular velocity. 
        \item Calculate the initial and final rotational kinetic energy.
        \item Use the right-hand rule to determine the direction of the 
            skaters angular momentum vector when spinning counterclockwise
            when viewed from above.
    \end{enumerate}
\end{exercise}

\subsection*{Exercise 2}
\begin{exercise}
    A $6kg$ bowling ball with radius $r=0.3m$ starts from rest and rolls
    down a $1.5m$ tall incline without slipping.
    \begin{enumerate}[label=(\alph*)]
        \item Find the moment of inertia of the bowling ball.
        \item Determine the final rotational kinetic energy at the 
            bottom of the incline.
        \item what is the final velocity and angular velocity of the sphere.
    \end{enumerate}
\end{exercise}
\end{document}
